\documentclass[
a4paper,
10pt,
twoside,
% prd,
% aps,
% nofootinbib,
% superscriptaddress,
% floatfix,
% preprintnumbers,
]{article}

% MUST BE RUN WITH LUALATEX

\usepackage{preamble}
\usepackage{titleinfo}

\geometry{ % Set document margins
	top			= 1cm,
	bottom		= 2cm,
	left		= 1cm,
	right		= 1cm
}

\newcommand{\mcols}{2} % Choose number of columns (>= 1)


\bibSetup{refs.bib} % Give references file 
% ===== Format headers & footers =====

\pagestyle{fancy}
\fancyhf{}
\fancyhead[LE,RO]{B. Henke}
\fancyhead[LO]{\headertitle\hspace{0.5cm}\textit{PHY852}}
\fancyhead[RE]{\textit{PHY852}\hspace{0.5cm}\headertitle}
\fancyfoot[RE,LO]{\thepage}

\begin{document}
% \tableofcontents
\titleinf
\maketitle
\startmcols

\section{Sakurai 4.4}
\subsection{}
\begin{align}
	\mathcal{Y}_{0}^{1/2,1/2} &= \begin{pmatrix}
		Y_0^0 \\ 0
	\end{pmatrix},\\
	&= \frac{1}{\sqrt{4\pi}} \begin{pmatrix}
		1 \\ 0
	\end{pmatrix}.
\end{align}
\subsection{}
\begin{align}
	(\bm{\sigma}\vdot\vb{x})\mathcal{Y}_{0}^{1/2,1/2} &= \frac{r}{\sqrt{4\pi}}\begin{pmatrix}
		\cos\theta \\ e^{i\phi}\sin\theta
	\end{pmatrix},\\
	&= \frac{r}{\sqrt{3}}\begin{pmatrix}
		Y_1^0 \\ -\sqrt{2}Y_1^1
	\end{pmatrix},\\
	&= -r \mathcal{Y}_{1}^{1/2,1/2}.
\end{align}
Here, $j$ and $m$ remained the same, but $l$ increased from $0$ to $1$.

\subsection{}
Since $\vb{S}\vdot\vb{x}$ is a scalar under rotations, it can only connect states with $j' = j$ and $m'=m$ (Wigner-Ekhart Theorem).
However, the operator connects states of opposite parity.
This is done by changing $l$ by one unit.

\section{Sakurai 4.5}
The symmetry of $V$ is determined by $\vb{S}\vdot\vb{p}$, which is a pseudoscalar.
Therefore the matrix element $C_{n'l'j'm'}$ is zero unless $j'=j$, $m'=m$, and $l'=l\pm 1$.
The radial wave functions go as $r^l$, so, since $V(\vb{x})\propto \delta^{3}(\vb{x})$, the matrix element will only be nonzero for states with $l=0$.
Consequently, this interaction only connects $S_{1/2}$ and $P_{1/2}$ states.

\section{Sakurai 4.7}
\subsection{}
\begin{align}
	\psi_{p}(x,t) &= e^{i(p \cdot x - Et)/\hbar}.\\
	\psi_{p}^*(x,-t) &= e^{-i(p \cdot x + Et)/\hbar},\\
	&= e^{i(-p \cdot x - Et)/\hbar},\\
	&= \psi_{-p}(x,t). \qquad \checkmark
\end{align}

\subsection{}
\subsubsection*{Option 1}
\begin{align}
	\chi_+(n) &= \begin{pmatrix}
			e^{-i\phi/2}\cos\frac{\theta}{2}\\
			e^{i\phi/2}\sin\frac{\theta}{2}
		\end{pmatrix}.\\
	i\sigma_2\chi_+^*(n) &= \begin{pmatrix}
			0 & -1\\ 1 & 0
		\end{pmatrix}\begin{pmatrix}
			e^{i\phi/2}\cos\frac{\theta}{2}\\
			e^{-i\phi/2}\sin\frac{\theta}{2}
		\end{pmatrix},\\
		&= \begin{pmatrix}
			-e^{-i\phi/2}\sin\frac{\theta}{2}\\
			e^{i\phi/2}\cos\frac{\theta}{2}
		\end{pmatrix}.\qquad \checkmark
\end{align}

\subsubsection*{Option 2}
\begin{align}
	\chi(n) &= e^{-i\sigma_3\phi/2}e^{-i\sigma_2\theta/2}.\\
	-i\sigma_2 \chi^*(n) &= -i\sigma_2 e^{i\sigma_2\theta/2}e^{i\sigma_3\phi/2},\\
	&= e^{-i\sigma_2\pi/2}e^{i\sigma_2\theta/2}e^{i\sigma_3\phi/2}. \qquad \checkmark
\end{align}


\section{Sakurai 4.11}
\begin{align}
	\ev*{\vu{L}} &= \ev*{\vu{L}}{E},\\
	&= \ev*{\hat{\Theta}\vu{L}\hat{\Theta}^{-1}}{\tilde{E}},\\
	&= -\ev*{\vu{L}}{\tilde{E}},\\
	&= -\ev*{\vu{L}}{E}.\\
	\therefore \ev*{\vu{L}} &= 0.
\end{align}
\begin{align}
	\ip*{\vb{x},\tilde{E}} &= e^{-i\delta}\ip*{\vb{x},E},\\
	&= e^{-i\delta} \sum_{lm} F_{lm}(r)Y_l^m(\theta,\phi).\\
	\ip*{\vb{x},\tilde{E}} &= \ip*{E}{\vb{x}},\\
	&= \sum_{lm} F_{lm}^*(r)(Y_l^m)^*(\theta,\phi),\\
	&= \sum_{lm} F_{l,-m}^*(r)(-1)^{-m}Y_l^{-m}(\theta,\phi),\\
	\therefore F_{lm}(r) &= (-1)^m e^{-i\delta} F_{l,-m}^*(r).
\end{align}
% \printbib
\stopmcols

\end{document}