\documentclass[
a4paper,
10pt,
twoside,
% prd,
% aps,
% nofootinbib,
% superscriptaddress,
% floatfix,
% preprintnumbers,
]{article}

% MUST BE RUN WITH LUALATEX

\usepackage{preamble}
\usepackage{titleinfo}

\geometry{ % Set document margins
	top			= 1cm,
	bottom		= 2cm,
	left		= 1cm,
	right		= 1cm
}

\newcommand{\mcols}{2} % Choose number of columns (>= 1)


\bibSetup{refs.bib} % Give references file 
% ===== Format headers & footers =====

\pagestyle{fancy}
\fancyhf{}
\fancyhead[LE,RO]{B. Henke}
\fancyhead[LO]{\headertitle\hspace{0.5cm}\textit{PHY852}}
\fancyhead[RE]{\textit{PHY852}\hspace{0.5cm}\headertitle}
\fancyfoot[RE,LO]{\thepage}

\begin{document}
% \tableofcontents
\titleinf
\maketitle
\startmcols

\section{Perturbative Hydrogen Fine Structure}\label{sec: I}
\subsection{}\label{ssec: Ia}
\begin{align}
	\Delta_{l}^{(1)} &= \mel*{nlm}{H_D'}{nlm},\\
	&= \int \dd[3]{r} \frac{\pi\hbar^3\alpha}{2m^2c} \delta^{(3)}(\vb{r}) \abs{\Psi_{nl}^{m}(\vb{r})}^2,\\
	&= \frac{\pi\hbar^3\alpha}{2m^2c} \abs{\Psi_{nl}^{m}(\vb{0})}^2,\\
	&= 0, \qquad \text{when $l\neq 0$}.
\end{align}

\subsection{}\label{ssec: Ib}

\begin{align}
	\Delta_{0}^{(1)} &= \mel*{n00}{H_D'}{n00},\\
	&= \frac{\pi\hbar^3\alpha}{2m^2c} \abs{\Psi_{n0}^{0}(\vb{0})}^2,\\
	&= \frac{\pi\hbar^3\alpha}{2m^2c} \abs{R_{n0}(0)\mathcal{Y}_0^{0}(\phi,\theta)}^2,\\
	&= \frac{\pi\hbar^3\alpha}{2m^2c} \frac{1}{4\pi}\abs{\frac{2}{n a_0^{3/2}}}^2,\\
	&= \frac{\pi\hbar^3\alpha}{2m^2c} \frac{1}{4\pi}\frac{4}{n^2 a_0^3},\\
	&= \frac{\alpha^4 mc^2}{2n^3}. \qquad (a_0 = \tfrac{\hbar}{mc\alpha}).
\end{align}

\subsection{}\label{ssec: Ic}
\begin{align}
	E_{FS}^{(1)} &= E_{K}^{(1)} + \Delta_{0}^{(1)},\\
	&= E_n^{(0)}\alpha^2\left(\frac{1}{n(l+1/2)}-\frac{3}{4n^2}\right) + \frac{\alpha^4 mc^2}{2n^3},\\
	&\quad E_n^{(0)}=-\frac{1}{2}mc^2 \frac{\alpha^2}{n^2}.\\
	&= E_n^{(0)}\alpha^2\left(\frac{1}{n}-\frac{3}{4n^2}\right).\\
	E_{n0,j=1/2}^{(1)} &= E_n^{(0)}\alpha^2\left(\frac{1}{n}-\frac{3}{4n^2}\right). \qquad \checkmark
\end{align}

\section{Sakurai 5.6}\label{sec: II}
The energy eigenstates of the infinite square well in two dimensions are:
\begin{equation}
	\psi(\vb{r}) = \frac{2}{L}\sin(x\frac{n_x\pi}{L})\sin(y\frac{n_y\pi}{L}).
\end{equation}
The energies of these states are
\begin{equation}
	E_{n_x,n_y} = \frac{\hbar^2}{2m} \left(\frac{(n_x+n_y)\pi}{L}\right)^2
\end{equation}
For a total $N = n_x + n_y$, there are $N+1$ states with the same energy.
The ground state is $n_x = n_y = 1$, and the first excited state is $n_x = 2$ and $n_y = 1$ or vice versa.
Denote the energy states as $\ket*{n_x,n_y}$.

\begin{align}
	\mel*{1,1}{\hat{V_1}}{1,1} &= \lambda\mel*{1,1}{\hat{x}\hat{y}}{1,1},\\
	&= \lambda \frac{L^2}{4}.
\end{align}
\begin{align}
	V &= \lambda \begin{pmatrix}
		\mel*{1,2}{xy}{1,2} & \mel*{1,2}{xy}{2,1}\\
		\mel*{2,1}{xy}{1,2} & \mel*{2,1}{xy}{2,1}
	\end{pmatrix},\\
	&= \frac{\lambda L^2}{4\pi^4} \begin{pmatrix}
		\pi^4 & 1024/81\\
		1024/81 & \pi^4
	\end{pmatrix},\\
	\therefore \Delta_{1\pm}^{(1)} &= \frac{1}{81}(81\pi^4 \pm 1024).
\end{align}
The eigenvectors (zeroth-order energy eigenkets) are
\begin{align}
	\ket*{\Delta_{1\pm}^{(1)}} &= \frac{1}{\sqrt{2}} \left(\ket*{1,2}\pm\ket*{2,1}\right).
\end{align}

\section{Sakurai 5.19}\label{sec: III}

\begin{align}
	H &= \begin{pmatrix}
		E_2 + h\delta & 3ea_0\mathcal{E}\\
		3ea_0\mathcal{E} & E_2
	\end{pmatrix}.
\end{align}
The eigenvalues, $E$, are
\begin{align}
	E_\pm &= -\frac{h \delta}{2} \pm \frac{1}{2}\sqrt{h^2\delta^2 + 36e^2\mathcal{E}^2 a_0^2} + E_2.
\end{align}

Taking the limit where $\delta\ll 3ea_0\mathcal{E}$,
\begin{align}
	E_\pm &\approx -\frac{h \delta}{2} \pm 3ea_0\mathcal{E} + E_2,
\end{align}
which is indeed linear in $\mathcal{E}$.

In the limit where $\delta\gg 3ea_0\mathcal{E}$,
\begin{align}
	E_\pm &= -\frac{h \delta}{2} \pm \frac{1}{2}\left(h\delta + \frac{9e^2a_0^2\mathcal{E}^2}{2h\delta}\right) + E_2,
\end{align}
which is quadratic in $\mathcal{E}$.

% \printbib
\stopmcols

\end{document}