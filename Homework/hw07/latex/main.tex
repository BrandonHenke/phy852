\documentclass[
a4paper,
10pt,
twoside,
% prd,
% aps,
% nofootinbib,
% superscriptaddress,
% floatfix,
% preprintnumbers,
]{article}

% MUST BE RUN WITH LUALATEXMK

\usepackage{preamble}
\usepackage{titleinfo}

\geometry{ % Set document margins
	top			= 1cm,
	bottom		= 2cm,
	left		= 1cm,
	right		= 1cm
}

\newcommand{\mcols}{2} % Choose number of columns (>= 1)


\bibSetup{refs.bib} % Give references file 
% ===== Format headers & footers =====

\pagestyle{fancy}
\fancyhf{}
\fancyhead[LE,RO]{B. Henke}
\fancyhead[LO]{\headertitle\hspace{0.5cm}\textit{PHY852}}
\fancyhead[RE]{\textit{PHY852}\hspace{0.5cm}\headertitle}
\fancyfoot[RE,LO]{\thepage}

\begin{document}
% \tableofcontents
\titleinf{Problem Set 7}
\maketitle
\startmcols

\section{Sakurai 7.3}\label{sec: I}


\subsection{} \label{ssec: IA}


The energy for one harmonic oscillator is
\begin{align}
	E_n &= \hbar \omega \left(n+\frac{1}{2}\right),
\end{align}
Where $\omega$ is the angular frequency, $\sqrt{k/m}$.
If one ignores the spin states, of the particles, the fermions cannot occupy the same state.
Hence, the first particle will be in the ground state ($n=0$), the second particle will be in the first excited state ($n=1$), etc.
Therefore the total energy is
\begin{align}
	E_{tot} &= \sum_{n=0}^{N-1} \hbar\omega\left(n + \frac{1}{2}\right),\\
	&= \frac{N^2}{2}\hbar\omega.
\end{align}

However, if one considers the spin state of each particle, then each energy level is doubly degenerate.
Two particles can occupy the same state, given they have opposing spin states.
Thus, the total energy in this case is
\begin{align}
	E_{tot} &= \sum_{n=0}^{\lfloor(N-1)/2\rfloor} 2\hbar\omega\left(n+\frac{1}{2}\right)-\left\{
		\begin{array}{rl}
			0 & N\;\text{even}\\
			\frac{N}{2} & N\;\text{odd}
		\end{array}\right.,\\
	&= \hbar\omega\left(\left\lfloor\frac{N-1}{2}\right\rfloor+1\right)^2-\left\{\begin{array}{rl}
		0 & N\;\text{even}\\
		\frac{N}{2} & N\;\text{odd}
	\end{array}\right.,\\
	&= \left\{\begin{array}{rl}
		\frac{N^2}{4}\hbar\omega & N\;\text{even}\\
		\frac{N^2+1}{4}\hbar\omega & N\;\text{odd}
	\end{array}\right..
\end{align}

The Fermi energy is the energy of the highest occupied level.
Hence, for the case where spin is ignored, it is $\hbar\omega(N-1 + 1/2)$.
For the case where spin is not ignored, the Fermi energy is
\begin{align}
	E_F &= \hbar\omega\left(\left\lfloor \frac{N-1}{2} \right\rfloor + \frac{1}{2}\right)
\end{align}

\subsection{} \label{ssec: IB}


For large $N$, $E_{tot} = N^2\hbar\omega/4$, and $E_F = N\hbar\omega/2$.


\section{Sakurai 7.7}\label{sec: II}


\subsection{}\label{ssec: IIA}


\begin{align}
	i.\qquad&\nonumber\\
	\ket{\psi} &= \ket{+}\ket{+}\ket{+},\\
	ii.\qquad&\nonumber\\
	\ket{\psi} &= \frac{1}{\sqrt{3}}(\ket{+}\ket{+}\ket{0} + \ket{+}\ket{0}\ket{+} + \ket{0}\ket{+}\ket{+})\\
	iii.\qquad&\nonumber\\
	\ket{\psi} &= \frac{1}{\sqrt{6}}(\ket{+}\ket{0}\ket{-} + \ket{+}\ket{0}\ket{-}\nonumber\\
	&\quad + \ket{0}\ket{+}\ket{-} + \ket{0}\ket{-}\ket{+}\nonumber\\
	&\quad + \ket{-}\ket{+}\ket{0} + \ket{-}\ket{0}\ket{+})
\end{align}
\subsection{}\label{ssec: IIB}
\begin{align}
	i.\qquad&\text{Not possible.}\nonumber\\
	ii.\qquad&\text{Not possible.}\nonumber\\
	iii.\qquad&\nonumber\\
	\ket{\psi} &= \frac{1}{\sqrt{6}}(\ket{+}\ket{0}\ket{-} - \ket{+}\ket{0}\ket{-}\nonumber\\
	&\quad - \ket{0}\ket{+}\ket{-} + \ket{0}\ket{-}\ket{+}\nonumber\\
	&\quad + \ket{-}\ket{+}\ket{0} - \ket{-}\ket{0}\ket{+})
\end{align}


\section{Sakurai 7.11}\label{sec: III}


\subsection{}\label{ssec: IIIA}
The position wave function in the infinite square well is
\begin{align}
	\psi_n(x) = \sqrt{\frac{2}{L}}\sin(\frac{n\pi x}{L}),
\end{align}
and the energy is given by
\begin{align}
	E_n = \frac{n^2 \pi^2 \hbar^2}{2mL^2}.
\end{align}
In the triplet state, the spin state is symmetric, so the position state must be antisymmetric.
Hence, the ground state is
\begin{align}
	\psi(x_1,x_2) = \frac{1}{\sqrt{2}} \left[\psi_1(x_1)\psi_2(x_2)-\psi_1(x_2)\psi_2(x_1)\right]
\end{align}
The energy of this state is
\begin{align}
	\hat{H}\psi(x_1,x_2) &= \left[\frac{\pi^2\hbar^2}{2mL^2} + \frac{4\pi^2\hbar^2}{2mL^2}\right]\psi(x_1,x_2),\\
	&= \frac{5\pi^2\hbar^2}{2mL^2}\psi(x_1,x_2).\\
	\therefore E_{gs} &= \frac{5\pi^2\hbar^2}{2mL^2}.
\end{align}


\subsection{}\label{ssec: IIIB}


The singlet state is antisymmetric, so the spatial state must be symmetric.
Hence the ground state is
\begin{align}
	\psi(x_1,x_2) &= \psi_1(x_1)\psi_1(x_2)
\end{align}
The energy of this state is
\begin{align}
	\hat{H}\psi(x_1,x_2) &= \left[\frac{\pi^2\hbar^2}{2mL^2} + \frac{\pi^2\hbar^2}{2mL^2}\right]\psi(x_1,x_2),\\
	&= \frac{\pi^2\hbar^2}{mL^2}\psi(x_1,x_2).\\
	\therefore E_{gs} &= \frac{\pi^2\hbar^2}{mL^2}.
\end{align}


\subsection{}\label{ssec: IIIC}


The state is subject to a potential
\begin{align}
	V = -\lambda \delta(x_1 - x_2). \qquad (\lambda > 0)
\end{align}
For the antisymmetric spatial state, this potential has no effect, because the Pauli exclusion principle doesn't allow for the particles to be at the same position.
This rules out the triplet state.

For the singlet state, however, the two particles can be in the same position.
Then, the energy shift is given by
\begin{align}
	\Delta E &= \ev{V},\\
		&= -\lambda \iint\dd{x_1}\dd{x_2} \psi_1^2(x_1)\psi_1^2(x_2)\delta(x_1-x_2),\\
		&= \int\dd{x} \psi_1^4(x),\\
		&= -\frac{3\lambda}{2L}.
\end{align}

% \printbib
\stopmcols

\end{document}