\documentclass[
a4paper,
10pt,
twoside,
% prd,
% aps,
% nofootinbib,
% superscriptaddress,
% floatfix,
% preprintnumbers,
]{article}

% MUST BE RUN WITH LUALATEXMK

\usepackage{preamble}
\usepackage{titleinfo}

\geometry{ % Set document margins
	top			= 1cm,
	bottom		= 2cm,
	left		= 1cm,
	right		= 1cm
}

\newcommand{\mcols}{2} % Choose number of columns (>= 1)


\bibSetup{refs.bib} % Give references file 
% ===== Format headers & footers =====

\pagestyle{fancy}
\fancyhf{}
\fancyhead[LE,RO]{B. Henke}
\fancyhead[LO]{\headertitle\hspace{0.5cm}\textit{PHY852}}
\fancyhead[RE]{\textit{PHY852}\hspace{0.5cm}\headertitle}
\fancyfoot[RE,LO]{\thepage}

\begin{document}
% \tableofcontents
\titleinf{Problem Set 8}
\maketitle
\startmcols

\section{Carbon Atom}\label{sec: I}
\subsection{} \label{ssec: IA}

The allowed values for the total angular momentum, added between two angular momenta, are $\{l_1+l_2,l_1+l_2-1,l_1+l_2-2,\dots,\abs{l_1-l_2}\}$.
Starting with $L$, $l_1 = l_2 = 1$ ($p$ orbital).
Hence,
\begin{align}
	L \in \{2,1,0\}.
\end{align}
Next, for the spin, $s_1 = s_2 = 1/2$.
Therefore,
\begin{align}
	S \in \{1,0\}.
\end{align}

Finding $J$, is a little more complicated.
Electrons are fermions, so the states they can be in must be antisymmetric.
Thus, only combinations of $S$ and $L$ where one is symmetric and the other is antisymmetric are allowed.
The $L = 0$ and $L = 2$ states are symmetric, while the $L = 1$ state is antisymmetric.
Similarly, the $S = 1$ state is symmetric, while the $S = 0$ state is antisymmetric.
Hence the following table gives the allowed states
\begin{align}
	\begin{array}{ccc}
		S & L & J\\\hline
		0 & 2 & 2\\
		0 & 0 & 0\\
		1 & 1 & 0,1,2
	\end{array}\nonumber
\end{align}

Next, for each allowed value for $J$, there are $2J+1$ states such that their total angular momentum is $J$.
Therefore, the total number of states allowed is $5 + 1 + 1 + 3 + 5 = 15$ states.

\subsection{} \label{ssec: IB}

In the $2p$ orbital, the allowed values of $m_l$ are $\{-1,0,1\}$, and the allowed values for $m_s$ are $\{-1/2,1/2\}$.
Hence there are $6$ states for one electron in this orbital.
If the state of the two electrons is to be antisymmetric, then there are $6$ possible states for the first electron, but only $5$ for the second electron (one can't pick the same state for both).
However, this double counts since the order of picking states for each electron shouldn't matter.
Hence the total number of states allowed for the two electron system is $6\cdot 5 /2 = 15$, which agrees with \ref{sec: I}-\ref{ssec: IA}.






\section{Two-particle Operator}\label{sec: II}


\begin{align}
	\{ab\mid V\mid cd\}
		&= \frac{1}{2} (\bra{ab}-\bra{ba})V(\ket{cd}-\ket{dc}),\\
		&= \frac{1}{2} \left(\mel*{ab}{V}{cd}-\mel*{ab}{V}{dc}\right)\nonumber\\
			&\quad-\frac{1}{2}\left(\mel*{ba}{V}{cd}-\mel*{ba}{V}{dc}\right),\\
		&= \frac{1}{2} \left(\mel*{ab}{V}{cd}-\mel*{ab}{V}{dc}\right)\nonumber\\
			&\quad+\frac{1}{2} \left(\mel*{ab}{V}{cd}-\mel*{ab}{V}{dc}\right),\\
		&= \mel*{ab}{V}{cd}-\mel*{ab}{V}{dc},\\
		&= \mel*{ab}{V}{cd}-\mel*{ba}{V}{cd}.
\end{align}


\section{Non-interacting Fermions in a 1D Harmonic Oscillator Potential}\label{sec: III}
\subsection{}\label{ssec: IIIA}


\begin{align}
	\hat{H}
		&= \sum_{n} \hbar\omega\left(\hat{a}_n^\dagger \hat{a}_n + \frac{1}{2}\right),\\
	\rightarrow \hat{\mathcal{H}}
		&= \sum_{kl} \hbar\omega\left(l + \frac{1}{2}\right)\delta_{kl}\hat{a}_k^\dagger \hat{a}_lcols,\\
		&= \sum_{n} \hbar\omega\left(n + \frac{1}{2}\right)\hat{a}_n^\dagger \hat{a}_n.
\end{align}


\subsection{}\label{ssec: IIIB}

The ground state is the state that is annihilated by $\hat{a}_j$ for $j \in \N$.
Hence
\begin{align}
	\ket{\psi_{gs}} &= \ket{0}.
\end{align}
(I'm not really sure what the desired answer to this one is since all sources I've found say that 
it's just this.)


\subsection{}\label{ssec: IIIC}


\begin{align}
	\hat{\psi}(\vb{r}) &= \sum_n \ip*{\vb{r}}{n}\hat{a}_n,\\
	\hat{\psi}^\dagger(\vb{r}) &= \sum_n \hat{a}_n^\dagger \ip*{n}{\vb{r}}.
\end{align}


\subsection{}\label{ssec: IIID}


\begin{align}
	\rho(\vb{x},\vb{x}')
		&= \ev{\hat{\psi}^\dagger(\vb{x})\hat{\psi}(\vb{x}')},\\
		&= \sum_{mn}\ev{\hat{a}_m^\dagger \ip*{m}{\vb{x}}\ip*{\vb{x}'}{n}\hat{a}_n},\\
		&= \sum_{mn}\ip*{m}{\vb{x}}\ip*{\vb{x}'}{n} \ev{\hat{a}_m^\dagger \hat{a}_n}{k},\\
		&= \sum_m \ip*{m}{\vb{x}}\ip*{\vb{x}'}{m}.
\end{align}


\section{Second Quantization of Operators}\label{sec: IIII}
\subsection{}\label{ssec: IIIIA}


\begin{align}
	\hat{\mathcal{K}} &= \sum_{mn}\mel*{l_m}{\hat{K}}{l_n}\hat{b}_{m}^\dagger \hat{b}_{n}.
	\label{eq: gen_op}
\end{align}
\begin{align}
	&\hat{\rho}(\vb{r}) = \sum_{n}^N \delta(\vb{r}-\hat{\vb{r}}_n),\\
	\rightarrow &\hat{\varrho}(\vb{r}) = \nonumber\\
	&\sum_{m_s'm_s''}\iint \dd{\vb{x}'}\dd{\vb{x}''} \mel*{\vb{x}'m_s'}{\delta(\vb{r}-\hat{\vb{r}})}{\vb{x}''m_s''}\hat{\psi}_{m_s'}^\dagger(\vb{x}') \hat{\psi}_{m_s''}(\vb{x}''),\\
	&= \sum_{m_s'm_s''}\iint \dd{\vb{x}'}\dd{\vb{x}''} \delta(\vb{r}-\vb{x}'')\ip*{\vb{x}'m_s'}{\vb{x}''m_s''}\hat{\psi}_{m_s'}^\dagger(\vb{x}') \hat{\psi}_{m_s''}(\vb{x}''),\\
	&= \sum_{m_s}\hat{\psi}_{m_s}^\dagger(\vb{r}) \hat{\psi}_{m_s}(\vb{r}).
\end{align}


\subsection{}\label{ssec: IIIIB}


In a similar manner to section \ref{sec: IIII}-\ref{ssec: IIIIA},
\begin{align}
	&\mel*{\vb{x}'m_s'}{\hat{\vb{p}}\delta(\vb{r}-\hat{\vb{r}})+\delta(\vb{r}-\hat{\vb{r}})\hat{\vb{p}}}{\vb{x}'m_s'}\nonumber\\
		&\qquad= \mel*{\vb{x}'m_s'}{\hat{\vb{p}}\delta(\vb{r}-\hat{\vb{r}})}{\vb{x}''m_s''}\nonumber\\
		&\qquad\quad+\mel*{\vb{x}'m_s'}{\delta(\vb{r}-\hat{\vb{r}})\hat{\vb{p}}}{\vb{x}''m_s''},\\
		&\qquad= \frac{\hbar}{i}\partial_{\vb{x}'}\delta(\vb{r}-\vb{x}'')\ip*{\vb{x}'m_s'}{\vb{x}''m_s''}\nonumber\\
		&\qquad\quad+ \delta(\vb{r}-\vb{x}'')\frac{\hbar}{i}\partial_{\vb{x}'}\ip*{\vb{x}'m_s'}{\vb{x}''m_s''}
		\label{eq: 4b_mel}
\end{align}
Plugging \ref{eq: 4b_mel} into \ref{eq: gen_op} gives
\begin{align}
	\hat{\bm{\mathcal{j}}}(\vb{r}) &= \frac{\hbar}{2im} \sum_{m_s} \hat{\psi}_{m_s}^\dagger(\vb{r})\partial_{\vb{r}}\hat{\psi}_{m_s}(\vb{r}) - \partial_{\vb{r}}\hat{\psi}_{m_s}^\dagger(\vb{r})\hat{\psi}_{m_s}(\vb{r}).
\end{align}


\subsection{}\label{ssec: IIIIC}


Again, the proceedure is the same as in the last to parts of this problem.
Hence, the second quantization spin density operator is give by
\begin{align}
	\hat{\bm{\mathcal{S}}}(\vb{r}) &= \sum_{m_s} \frac{\hbar}{2}\sigma_{m_s} \hat{\psi}_{m_s}^\dagger(\vb{r})\hat{\psi}_{m_s}(\vb{r}).
\end{align}


% \printbib
\stopmcols

\end{document}