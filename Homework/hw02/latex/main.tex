\documentclass[
a4paper,
10pt,
twoside,
% prd,
% aps,
% nofootinbib,
% superscriptaddress,
% floatfix,
% preprintnumbers,
]{article}

% MUST BE RUN WITH LUALATEX

\usepackage{preamble}
\usepackage{titleinfo}

\geometry{ % Set document margins
	top			= 1cm,
	bottom		= 2cm,
	left		= 1cm,
	right		= 1cm
}

\newcommand{\mcols}{2} % Choose number of columns (>= 1)


\bibSetup{refs.bib} % Give references file 
% ===== Format headers & footers =====

\pagestyle{fancy}
\fancyhf{}
\fancyhead[LE,RO]{B. Henke}
\fancyhead[LO]{\headertitle\hspace{0.5cm}\textit{PHY852}}
\fancyhead[RE]{\textit{PHY852}\hspace{0.5cm}\headertitle}
\fancyfoot[RE,LO]{\thepage}

\begin{document}
% \tableofcontents
\titleinf
\maketitle
\startmcols

\section{Sakurai 5.1}\label{sec: I}

\begin{align}
	H_0 &= \frac{\hat{p}^2}{2m} + \frac{1}{2}\omega^2m^2\hat{x}^2,\\
	H &= H_0 + b\hat{x}.\\
	\Delta_0 &= \mel*{0^{(0)}}{b\hat{x}}{0^{(0)}} + \sum_{k \neq 0} \frac{2b^2 \abs{\mel*{1}{x}{0}}^2}{\hbar\omega - (2k+1)\hbar\omega},\\
	&= 0 + \sum_{k \neq 0} \frac{\hbar}{2m\omega}\frac{2b^2 \abs{\ip{k}{1}}^2}{\hbar\omega - (2k+1)\hbar\omega},\\
	&= \frac{\hbar}{2m\omega}\frac{2b^2}{-2\hbar\omega},\\
	&= -\frac{b^2}{2m\omega^2}
\end{align}
\begin{align}
	V(x) &\rightarrow \frac{1}{2}m \omega^2\left(x+\frac{b}{m\omega^2}\right)^2-\frac{b^2}{2m\omega^2}.\\
	x' &= x+\frac{b}{m\omega^2},\\
	H &= \frac{\hat{p}^2}{2m} + \frac{1}{2}\omega^2m^2\hat{x}'^2-\frac{b^2}{2m\omega^2}.
\end{align}
This is a simple harmonic oscillator with an energy shift of $-\frac{b^2}{2m\omega^2}$, so the perturbation theory result gives the exact energy shift.

\section{Sakurai 5.7}\label{sec: II}

\subsection{}\label{subsec: IIa}

This is just two independent simple harmonic oscillators, so the energy is
\begin{equation}
	E_{n_x,n_y} = \hbar\omega\left(n_x+n_y+1\right).
\end{equation}
Hence, the three lowest lying states ($\ket{0,0},\ket{1,0},$ and $\ket{0,1}$) have energies of $E \in \{ \hbar\omega, 2\hbar\omega, 2\hbar\omega \}$, respectively.
Hence, the first excited state is doubly degenerate.

\subsection{}\label{ssec: IIb}

For the ground state, there is no degeneracy, so the energy shift is
\begin{align}
	\Delta_0^{(0)} &= \delta m \omega^2 \mel*{0,0}{xy}{0,0},\\
	&= 0. \qquad (x\ket*{0}\propto \ket*{1})
\end{align}

For the first excited state:
\begin{align}
	V\ket*{l^{(0)}} = \Delta_1^{(1)}\ket*{l^{(0)}} &= \sum_{m} V\ket*{m^{(0)}}\ip*{m^{(0)}}{l^{(0)}}.
	\label{eq: 5.7 evp}
\end{align}
\begin{align}
	V &= \delta m \omega^2\begin{pmatrix}
		\mel*{0,1}{xy}{0,1} & \mel*{1,0}{xy}{0,1}\\
		\mel*{0,1}{xy}{1,0} & \mel*{1,0}{xy}{1,0}
	\end{pmatrix},\\
	&= \delta m \omega^2 \frac{\hbar}{2m\omega}\begin{pmatrix}
		0 & 1\\
		1 & 0
	\end{pmatrix},\\
	&= \delta \omega \frac{\hbar}{2}\begin{pmatrix}
		0 & 1\\
		1 & 0
	\end{pmatrix}.
\end{align}
Solving the eigenvalue problem (eq \ref{eq: 5.7 evp}), gives
\begin{equation}
	\Delta_1^{(1)} = \pm\delta\omega\hbar/2.
\end{equation}
The eigenkets, which correspond to the zero-th order energy eigenkets, are
\begin{align}
	\Delta_1^{(1)} = \pm\delta\omega\hbar/2:& \nonumber\\
	\ket*{l^{(0)}} &= \frac{1}{\sqrt{2}} (\ket*{0,1}\pm\ket*{1,0}).
\end{align}

\subsection{}\label{subsec: IIc}

The full Hamiltonian is
\begin{align}
	H &= \frac{p^2}{2m} + \frac{1}{2}m\omega^2 x^2 + \delta m \omega^2 xy,\\
	&= \frac{p^2}{2m} + \frac{1}{2}m\omega^2 (x^2+y^2+2xy\delta),\\
	&= \frac{p^2}{2m} + \frac{1}{2}m\omega^2 (x'^2+y'^2),
\end{align}
where $x' = \sqrt{1+\delta}\frac{x+y}{\sqrt{2}}$ and $y' = \sqrt{1-\delta}\frac{x-y}{\sqrt{2}}$.

From this, the energy of the energy eigenket $\ket*{n_{x'},n_{y'}}$ is
\begin{equation}
	E_{n_{x'},n_{y'}} = \left(n_{x'}+\frac{1}{2}\right)\hbar\omega(1+\delta)^{1/2}+\left(n_{y'}+\frac{1}{2}\right)\hbar\omega(1-\delta)^{1/2}.
\end{equation}
From this, Taylor expanding the energy of the energy eigenket $\ket*{0,0}$ is $\approx \hbar\omega - \delta^2\hbar\omega/8$.
The first term here, agrees with the perturbation theory result.
The Taylor expanded energies of the energy eigenkets $\ket*{1,0}$ and $\ket*{0,1}$ are $\approx (2+\delta/2)\hbar\omega$ and $\approx (2-\delta/2)\hbar\omega$, respectively.
These both agree with the perturbation theory results.

\section{Second Order Ground State Correction}\label{sec: III}

The second order correction to the ground state of the perturbation $V = \delta m \omega^2 xy$ is
\begin{align}
	\Delta_0 &= 0 + \sum_{k\neq 0} \delta^2m^2\omega^4\frac{\mel*{0,0}{xy}{k}\mel*{k}{xy}{0,0}}{\hbar\omega - E_k^{(0)}},\\
	&= \delta^2 m^2\omega^4\left(\frac{\hbar}{2m\omega}\right)^2\frac{1}{\hbar\omega - 3\hbar\omega},\\
	&= -\frac{\delta^2\hbar\omega}{8}.
\end{align}
This agrees with the second term in the taylor expansion of the energy for the exact ground state found in \ref{sec: II}-\ref{subsec: IIc}.

\section{Sakurai 5.16}\label{sec: IV}



% \printbib
\stopmcols

\end{document}