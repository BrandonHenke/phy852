\documentclass[
a4paper,
10pt,
twoside,
% prd,
% aps,
% nofootinbib,
% superscriptaddress,
% floatfix,
% preprintnumbers,
]{article}

% MUST BE RUN WITH LUALATEXMK

\usepackage{preamble}
\usepackage{titleinfo}

\geometry{ % Set document margins
	top			= 1cm,
	bottom		= 2cm,
	left		= 1cm,
	right		= 1cm
}

\newcommand{\mcols}{2} % Choose number of columns (>= 1)


\bibSetup{refs.bib} % Give references file 
% ===== Format headers & footers =====

\pagestyle{fancy}
\fancyhf{}
\fancyhead[LE,RO]{B. Henke}
\fancyhead[LO]{\headertitle\hspace{0.5cm}\textit{PHY852}}
\fancyhead[RE]{\textit{PHY852}\hspace{0.5cm}\headertitle}
\fancyfoot[RE,LO]{\thepage}

\begin{document}
% \tableofcontents
\titleinf{Problem Set 9}
\maketitle
\startmcols

\section{Sakurai 6.3}\label{sec: I}

\begin{align}
	q &= 2k\sin(\frac{\theta}{2}),\\
	800\text{MeV} &= \frac{\hbar^2k^2}{2m},\qquad\text{(from figure)}\\
	k &= 6.11\text{fm}^{-1}.\\
	qa &= (4.49,7.73,10.9,...),\\
	\theta &= (7.7,14.25,20.95)^\circ,\\
	a &= (5.47, 5.10, 4.91)\text{fm}.\\
	1.4 A^{1/3} &= 4.79\text{fm}.
\end{align}
My estimates are a bit large.

\section{Scattering Off of a Spherically-Symmetric Potential}\label{sec: II}
\subsection{}\label{ssec: IIa}

\begin{align}
	f
		&= -\frac{m}{2\pi\hbar^2}\int\dd[3]{x} e^{-i \bm{q}\cdot \bm{x}}V_0 e^{-\frac{r^2}{2a^2}},\\
		&= -\frac{m V_0}{2\pi\hbar^2}\left(\int\dd{x} e^{-i q_x x}e^{-\frac{x^2}{2a^2}}\right)^3,\\
		&= -\frac{\sqrt{2\pi} a^3 m V_0}{\hbar^2}  e^{-3a^2 q^2/2}.\\
	\dv{\sigma}{\Omega}
		&= \abs{f}^2,\\
		&= \frac{2\pi a^6 m^2 V_0^2}{\hbar^4} e^{-3a^2 q^2}.
\end{align}


\subsection{}\label{ssec: IIb}

\begin{align}
	\sigma
		&= \int \dv{\sigma}{\Omega} \dd{\Omega},\\
		&= \int_0^{2\pi} \dd{\phi} \int_{-1}^{1} \dv{\sigma}{\Omega} \dd{\cos\theta},\\
		&= \frac{4\pi^2 a^6 m^2 V_0^2}{\hbar^4} e^{-3a^2} \int_{-1}^{1} e^{q^2} \dd{\cos\theta},\\
		&= \frac{4\pi^2 a^6 m^2 V_0^2}{\hbar^4} \int_{-1}^{1} e^{-6a^2k^2(1-\cos\theta)} \dd{\cos\theta},\\
		&= \frac{2\pi^2 a^4 m^2 V_0^2(1 - e^{-12 a^2 k^2})}{3 k^2 \hbar^4}.
\end{align}


\section{Scattering Off of a "Separable Potential"}\label{sec: III}
\subsection{}\label{ssec: IIIa}

\begin{align}
	T
		&= \op{v}(1+\ev{G_0}{v}+\ev{G_0}{v}^2+\dots),\\
		&= \op{v}\sum_{n=0}^\infty \ev{G_0}{v}^n,\\
		&= \frac{\op{v}}{1-\ev{G_0}{v}},\\
	\ev{G_0}{v}
		&= \frac{2m}{\hbar^2} \int \dd[3]{\bm{q}} \frac{\abs{v(\bm{k}')}^2}{k^2 - q^2 + i\epsilon}.\\
	T(\bm{k}',\bm{k})
		&= \frac{v(\bm{k}')v^*(\bm{k})}{1-\ev{G_0}{v}}.
\end{align}

\subsection{}\label{ssec: IIIb}

The Fourier transform of a constant is a dirac delta function:
\begin{align}
	\mel*{\bm{r}'}{V}{\bm{r}}
		&= g\int\dd[3]{\bm{k}'}\dd[3]{\bm{k}} \frac{e^{i\bm{k}'\cdot\bm{r}'}e^{-i\bm{k}\cdot\bm{r}}}{(2\pi)^3},\\
		&= (2\pi)^3 g \delta(\bm{r}) \delta(\bm{r}').
\end{align}

\subsection{}\label{ssec: IIIc}


\begin{align}
	T(0,0) = a
		&= \frac{g}{1-\ev{G_0}{v}},\\
	g
		&= a(1-\ev{G_0}{v}),\\
		&= a\left(1 - \frac{2m}{\hbar^2}g \int_0^\Lambda \dd[3]{\bm{q}} \frac{1}{- q^2 + i\epsilon}\right),\\
		&= a\left(1 - \frac{8\pi m g}{\hbar^2}\Lambda\right),\\
	g(\Lambda)
		&= \frac{a}{1-\frac{8\pi m a \Lambda}{\hbar^2}}.
\end{align}


\subsection{}\label{ssec: IIId}


\begin{align}
	\dv{\Lambda} \frac{1}{T(\bm{k}',\bm{k})}
		&= -\frac{8\pi m}{\hbar^2} + \frac{8\pi m}{\hbar^2} \dv{\Lambda}\int_{0}^{\Lambda} \frac{q^2\dd{q}}{q^2-k^2},\\
		&= \frac{8\pi m}{\hbar^2} \frac{k^2}{\Lambda^2-k^2}.
\end{align}
For $k\ll\Lambda$, this expression goes to zero.


% \printbib
\stopmcols

\end{document}